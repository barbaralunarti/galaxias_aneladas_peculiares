% resumo em português
\setlength{\absparsep}{18pt} % ajusta o espaçamento dos parágrafos do resumo
\begin{resumo}
 
As galáxias peculiares aneladas (ring galaxies - RGs), fazem parte do grupo de galáxias irregulares e têm a aparência de um anel, resultado de uma fusão ou interação galáctica. O estudo desses objetos traz informações dos processos interativos que algumas galáxias passam, a taxa de formação de estrelas predominante nos anéis, dinâmica, trilhas evolutivas e populações estelares. Desta forma, a qualidade fotométrica está associada à redução de erros e tratamento dos dados, que podem distorcer as análises e levar a interpretações incorretas. Nesta pesquisa, estudamos a fotometria de três aberturas elípticas (auto, iso e petro) do Southern Photometric Local Universe Survey (S-PLUS), que possui um sistema de combinação de 12 filtros (bandas largas e estreitas), para nossa amostra de 116 RGs. Após a análise individual das aberturas para as galáxias de nossa amostra, concluímos que não são confiáveis e aplicáveis para todos os objetos, devido à sua estrutura perturbada e anelada. Desta forma, é necessária uma inspeção individual, separando os objetos que possuem fotometria ``confiável'' dos que não possuem. Para os objetos de qualidade fotométrica, podemos avaliar e comparar as medições de magnitudes e erros entre as aberturas, observando dispersões e confiabilidade de valores. Para as galáxias que as aberturas não abrangem toda sua extensão, temos apenas as informações de suas partes, não dela como um todo. Observamos também que para magnitudes maiores que 18 mag, há maior probabilidade de medições de erros para todas as aberturas.

 \textbf{Palavras-chave}:galáxias aneladas. fotometria. aberturas elípticas. S-PLUS.
\end{resumo}
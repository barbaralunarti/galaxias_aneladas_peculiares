% resumo em português
\setlength{\absparsep}{18pt} % ajusta o espaçamento dos parágrafos do resumo
\begin{resumo}
 
Apresentamos um estudo da fotometria de abertura elíptica do Southern Photometric Local Universe Survey (S-PLUS) para galáxias peculiares aneladas (ring galaxies - RGs). Estes objetos fazem parte do grupo de galáxias irregulares e têm a aparência de um anel, resultado de uma fusão ou efeitos da força potencial gravitacional em uma interação galáctica. O estudo desses objetos traz informações sobre os processos interativos e evolutivos que algumas galáxias passam e regiões onde novas estrelas estão se formando após esse cenário. Avaliamos a qualidade fotométrica das aberturas elípticas auto (raio de Kron), iso (área isofotal) e petro (raio de Petrosian) do S-PLUS, que possui um sistema de combinação de 12 filtros (5 bandas largas e 7 estreitas), para nossa amostra de 117 RGs, visto que possuem morfologia irregular, anéis extensos, interação em andamento ou efeitos de maré. Após a análise das aberturas para as galáxias de nossa amostra, concluímos que não são confiáveis e aplicáveis para todos os objetos, devido à sua estrutura perturbada e anelada. Desta forma, foi realizada uma inspeção individual, separando os objetos que possuem fotometria ``confiável'' dos que não possuem. Para os objetos de qualidade fotométrica, comparamos as medições de magnitudes e respectivos erros entre as aberturas, analisando as dispersões e confiabilidade de valores, bem como a característica da distribuição espectral de energia. Para as galáxias que as aberturas não abrangem toda sua extensão, temos apenas as informações de suas partes, não dela como um todo.

 \textbf{Palavras-chave}:galáxias aneladas. fotometria. aberturas elípticas. S-PLUS.
\end{resumo}
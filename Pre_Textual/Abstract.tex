% resumo em inglês
\setlength{\absparsep}{18pt} % ajusta o espaçamento dos parágrafos do resumo
\begin{resumo}[Abstract]
 \begin{otherlanguage*}{english}
  The peculiar ring galaxies (RGs) are part of the group of irregular galaxies and have the appearance of a ring, resulting from a galactic merger or interaction. The study of these objects provides information on the interactive processes that some galaxies undergo, the predominant star formation rate in the rings, dynamics, evolutionary paths, and stellar populations. Thus, photometric quality is associated with reducing errors and handling data, which can distort analyses and lead to incorrect interpretations. In this research, we studied the photometry of three elliptical apertures (auto, iso, and petro) from the Southern Photometric Local Universe Survey (S-PLUS), which has a combination system of 12 filters (broad and narrow bands), for our sample of 116 RGs. After the individual analysis of the apertures for the galaxies in our sample, we concluded that they are not reliable and applicable for all objects, due to their disturbed and ringed structure. Therefore, individual inspection is necessary, separating objects with "reliable" photometry from those that do not have it. For objects with photometric quality, we can evaluate and compare magnitude and error measurements between apertures, observing dispersions and reliability of values. For galaxies where the apertures do not cover their entire extent, we only have information about their parts, not about them as a whole. We also observed that for magnitudes greater than 18 mag, there is a higher probability of error measurements for all apertures.

   \vspace{\onelineskip}
 
   \noindent 
   \textbf{Keywords}: ring galaxies. photometry. elliptical apertures. S-PLUS.
 \end{otherlanguage*}
\end{resumo}
% resumo em inglês
\setlength{\absparsep}{18pt} % ajusta o espaçamento dos parágrafos do resumo
\begin{resumo}[Abstract]
 \begin{otherlanguage*}{english}
  We present a study of the elliptical aperture photometry from the Southern Photometric Local Universe Survey (S-PLUS) for peculiar ring galaxies (RGs). These objects belong to the group of irregular galaxies and have a ring-like appearance, resulting from a merger or the effects of gravitational potential force in a galactic interaction. The study of these objects provides information about the interactive and evolutionary processes that some galaxies undergo, and regions where new stars are forming after such events. We evaluated the photometric quality of the elliptical apertures auto (Kron radius), iso (isophotal area), and petro (Petrosian radius) from S-PLUS, which has a system combining 12 filters (5 broad bands and 7 narrow bands), for our sample of 117 RGs, considering their irregular morphology, extensive rings, ongoing interactions, or tidal effects. After analyzing the apertures for the galaxies in our sample, we concluded that they are not reliable or applicable for all objects due to their disturbed and ringed structure. Therefore, an individual inspection was carried out, separating objects with ``reliable'' photometry from those without. For the objects with reliable photometric quality, we compared the magnitude measurements and respective errors between the apertures, analyzing the dispersions and reliability of values, as well as the characteristic spectral energy distribution. For the galaxies where the apertures do not cover their entire extent, we only have information about their parts, not the whole.

   \vspace{\onelineskip}
 
   \noindent 
   \textbf{Keywords}: ring galaxies. photometry. elliptical apertures. S-PLUS.
 \end{otherlanguage*}
\end{resumo}
% resumo em inglês
\setlength{\absparsep}{18pt} % ajusta o espaçamento dos parágrafos do resumo
\begin{resumo}[Abstract]
 \begin{otherlanguage*}{english}
  We present a study of the elliptical aperture photometry from the Southern Photometric Local Universe Survey (S-PLUS) for peculiar ring galaxies (RGs). These objects belong to the group of irregular galaxies and have the appearance of a ring, resulting from a merger or the effects of gravitational potential force in a galactic interaction. The study of these objects provides information about the interactive and evolutionary processes that some galaxies undergo, the formation of new stars, and characteristics of stellar populations. We evaluated the photometric quality of the auto (Kron radius), iso (isophotal area), and petro (Petrosian radius) elliptical apertures from S-PLUS, which has a combination system of 12 filters (5 broadbands and 7 narrowbands), for our sample of 116 RGs, considering their irregular morphology, extensive rings, ongoing interaction, or tidal effects. After analyzing the apertures for the galaxies in our sample, we concluded that they are not reliable or applicable for all objects due to their disturbed and ringed structure. Thus, an individual inspection was carried out, separating the objects with "reliable" photometry from those without. For the objects with photometric quality, we compared the magnitude measurements and respective errors among the apertures, analyzing the dispersions and reliability of values, as well as the characteristics of the spectral energy distribution. For the galaxies where the apertures do not cover their entire extent, we only have information about their parts, not the whole.

   \vspace{\onelineskip}
 
   \noindent 
   \textbf{Keywords}: ring galaxies. photometry. elliptical apertures. S-PLUS.
 \end{otherlanguage*}
\end{resumo}
\chapter{Considerações Finais}

Neste trabalho realizamos o estudo da qualidade fotométrica do sistema de 12 filtros (combinação de
bandas estreitas e largas) do Southern Photometric Local Universe Survey (S-PLUS) para galáxias peculiares aneladas (RGs). As RGs fazem parte de um grupo das galáxias irregulares e possuem o formato de um anel. Sua estrutura perturbada e anelada, é resultado de colisões galácticas e interações entre forças de maré \cite{2010arnab}, um desafio para o estudo da fotometria, principalmente com aberturas elípticas. A qualidade da fotometria para estes objetos foi analisada nesta pesquisa para três aberturas elípticas utilizadas pelo S-PLUS: AUTO, ISO e PETRO. Os dados utilizados estão disponíveis pela busca ADQL do \emph{survey}, como imagens FITS Figure, dados fotométricos, astrométricos e parâmetros referentes às aberturas.

A análise da fotometria de abertura para cada RG é importante para uma melhor visualização das características e informações que podemos extrair desses objetos, visto sua natureza morfológica irregular. Para estes objetos irregulares é necessária uma inspeção individual, como foi realizada nesta pesquisa, separando as galáxias que podemos dizer que a detecção é de qualidade (fotometria confiável), pois compreende todo o objeto, dos que não podemos usar para análise de dados. Quando uma única abertura elíptica não abrange adequadamente o corpo e contorno da galáxia, o ideal é refazer a fotometria com uma técnica mais apropriada para o objeto, ajustando e testando os parâmetros para encontrar a melhor elipse que envolva a galáxia, ou a combinação de elipses, ou outros métodos e ferramentas como utilização de múltiplas elipses concêntricas (com diferentes tamanhos e excentricidades que podem capturar a variação na distribuição de luz em diferentes partes da galáxia), construir perfis de brilho superficial (entender como a intensidade da luz varia com a distância do centro da galáxia), dentre outros. Investigar a qualidade das detecções e como a fotometria é realizada, é fundamental para obtermos o mais próximo possível a característica e propriedades físicas dos objetos. Uma vez que isso não é feito, principalmente para galáxias peculiares, informações subestimadas ou superestimadas levam a conclusões incorretas.

Para nossa amostra com 117 galáxias, foram separados os objetos de fotometria ``confiável'' (a abertura abrange toda a extensão do objeto), e comparadas as medições fotométricas (magnitudes e erros) para as três aberturas nos doze filtros do S-PLUS. Cada abertura possui um método de medição do fluxo e respectivas incertezas (Seção \ref{sec:splus}), e para a maioria das galáxias separadas de nossa amostra, os valores são bem próximos entre si dentro da barra de erro. Desta forma, percebe-se que a medição dos valores das magnitudes possuem uma confiabilidade. Outro ponto importante e esperado, é observado aos objetos que possuem brilho fraco, apresentando maior probabilidade de erro na medição. Há também, poucas exceções de valores com medições de erro extrapoladas acima da média. 

Concluímos que para os objetos de morfologia irregular, a fotometria de abertura elíptica não é confiável e aplicável a todos de uma amostra. É necessária uma inspeção individual, analisando como a abertura é feita e para cada galáxia. Os valores de magnitudes que possuem barras de erro de medição significativos ou extrapolados, é importante que sejam excluídos da análise, porque não refletem as reais propriedades da galáxia, e a fotometria deve ser refeita. Com isso, para trabalhos futuros, ao avaliar como a detecção dos objetos é realizada e sua qualidade, podemos por exemplo, aplicar os dados do S-PLUS em modelos que interpretam as distribuições espectrais de energia das galáxias, para obter informações sobre a formação, massa, metalicidade, populações estelares, e atenuação da poeira, investigando os processos evolutivos das galáxias e propriedades do meio interestelar.


\chapter{Metodologia}

\section{Dados da amostra}

Inicialmente, as amostras foram selecionadas a partir de catálogos e atlas especializados em galáxias aneladas:

\begin{itemize}
    \item A catalogue of southern peculiar galaxies and associations de \citeonline{1987arpmadore} possui 274 galáxias na Categoria 6, classificadas com um anel bem definido ao redor dela e quaisquer galáxias morfologicamente parecidas. Os autores fazem uma breve descrição para cada objeto listado enfatizando aspectos relevantes, como a direção de um jato ou o número de companheiros aparentes, etc. Inspecionando um por um, 4 objetos foram retirados desta lista (3 galáxias espirais e 1 nebulosa planetária) de acordo com dados atuais de NASA/IPAC Extragalactic Database. Para estes 270 objetos considerados, todos foram classificados como \emph{peculiar ring galaxy} (PRG) neste estudo.
    \item New observations and a photographic atlas of polar-ring galaxies de \citeonline{1990AJ....100.1489W} possui um total de 157 galáxias e classificadas em quatro categorias: com anéis polares confirmados cinematicamente (categoria A), bons candidatos com base em sua aparência morfológica (categoria B), possíveis candidatos (categoria C) e objetos possivelmente relacionados (categoria D). Apenas as categorias A e B foram consideradas nesta pesquisa e classificadas como \emph{peculiar ring galaxy} (PRG), totalizando 33 galáxias.
    \item A new catalogue of polar-ring galaxies selected from the Sloan Digital Sky Survey (SDSS) de \citeonline{2011MNRAS.418..244M}, possui 275 objetos classificados como galáxias peculiares com anel polar conforme imagens do SDSS. Este catálogo complementa o de \citeonline{1990AJ....100.1489W} e é baseado nos resultados do projeto Galaxy Zoo. As galáxias de anel polar foram divididas em quatro categorias pelos autores: melhores candidatas, boas candidatas, objetos relacionados a anel polar e anéis muito inclinados à linha de visão (\emph{face on} - vistos quase de frente). Para esta pesquisa, foram considerados apenas os objetos melhores candidatos, bons candidatos e \emph{face on}, totalizando 222 galáxias e todas foram classificadas como \emph{peculiar ring galaxy} (PRG).
    \item Atlas and Catalog of Collisional Ring Galaxies de \citeonline{2009ApJS..181..572M}







\end{itemize}

Em um segundo momento, essas galáxias foram divididas em dois grupos distintos: as peculiares e as normais, seguindo rigorosamente as classificações previamente estabelecidas pelos autores dos catálogos e atlas consultados.

... complementar ...

Por fim, apenas as galáxias classificadas como peculiares foram consideradas para este estudo, representando assim nossa amostra primordial para o estudo da fotometria do S-PLUS.